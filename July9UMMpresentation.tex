\documentclass{beamer}
\usetheme{Warsaw}
\usecolortheme{wolverine}
%%wolverine is easier to see (especially on a screen) than dolphin

\usepackage{graphicx} %For jpg figure inclusion
\usepackage{times} %For typeface
\usepackage{epsfig}
\usepackage{color} %For Comments
\usepackage{float}
\usepackage{hyperref}
\usepackage{url}
\usepackage{parskip}


%% Elena's favorite green (thanks, Fernando!)
\definecolor{ForestGreen}{RGB}{34,139,34}
%% Henry's Color.
\definecolor{Teal}{RGB}{2,132,130}
%% Lemmon's Color.
\definecolor{Orange}{RGB}{200,127,0}
%% Emma's Color.
\definecolor{Blue}{RGB}{0,0,200}
% Uncomment this if you want to show work-in-progress comments
\newcommand{\comment}[1]{{\bf \tt  {#1}}}
% Uncomment this if you don't want to show comments
%\newcommand{\comment}[1]{}
\newcommand{\emcomment}[1]{\textcolor{ForestGreen}{\comment{Elena: {#1}}}}
\newcommand{\hfcomment}[1]{\textcolor{Teal}{\comment{Henry: {#1}}}}
\newcommand{\alcomment}[1]{\textcolor{Orange}{\comment{Lemmon: {#1}}}}
\newcommand{\escomment}[1]{\textcolor{Blue}{\comment{Emma: {#1}}}}
\newcommand{\todo}[1]{\textcolor{blue}{\comment{To Do: {#1}}}}
\newcommand{\clocode}[1]{{\texttt {#1}}}

\begin{document}
\title{Writing good software for teaching how to write good software}
\date{July 9, 2014}

\begin{frame}
\frametitle{HHMI lab meeting, July 9, 2014}
\emcomment{format better!}

{\centering
HHMI lab meeting, July 9, 2014 \par
}
Henry Fellows, Aaron Lemmon, Emma Sax\par
(supported by HHMI and UMM MAP grants)\par
Advisor: Elena Machkasova \par
\end{frame}

\begin{frame}[fragile]
\frametitle{Outline}
	\tableofcontents
\end{frame}


\section{Goals and scope of the project}

\begin{frame}[fragile]
\frametitle{The project}
Developing materials and tools for teaching an introductory CSci class using a new programming language called Clojure.

Clojure: appeared in 2007, is similar to the current language used in UMM introductory class, but provides additional benefits 
(support for execution on multiple processors, a wide community developing libraries, and used in industry)

Past work: 
\begin{itemize}
\item Joe Einertson, Stephen Adams (2012-13)
\item  Max Magnuson, Paul Schliep (2013-14)
\end{itemize}  
\end{frame}

\begin{frame}[fragile]
\frametitle{Learning goals of an introductory CSci class}
\begin{itemize}
\item Present programming as a structured activity
\item Develop good programming practices
\item Introduce successful techniques: testing, incremental development, refactoring, commenting 
\item Help students develop successful team-work strategies
\end{itemize}
\end{frame}

\begin{frame}[fragile]
\frametitle{Learning goals}
\emcomment{perhaps give an example of a "program": clear instructions, unclear instructions - for something simple? Recipe?}
{\it Programs should be written for people to read, and only incidentally for machines to execute} --	 
from "Structure and Interpretation of Computer Programs" by Abelson and Sussman
\end{frame}

\begin{frame}[fragile]
\frametitle{Elements of an introductory CSci class}
\begin{itemize}
\item A programming language
\item A text editor
\item Tools for testing, managing program code
\item Help in figuring out what went wrong
\item Challenging problems that students are interested in
\end{itemize}
\end{frame}

\begin{frame}[fragile]
\frametitle{Current elements for the Clojure-based class}
\begin{itemize}
\item A programming language: Clojure
\item A text editor: LightTable
\item Tools for testing, managing program code: testing library {\tt expectations}
\item Help in figuring out what went wrong: Clojure error messages (work-in-progress)
\item Challenging problems that students are interested in: graphical system (work-in-progress by Paul Schliep and Max Magnuson)
\end{itemize}
\emcomment{This is a lot of text, need to format, simplify; mention the starting state of the project}
\end{frame}

\section{Beginner-friendly software}

%\begin{frame}[fragile]
%\frametitle{What makes good code?}
%
%\begin{itemize}
%\item Good organization: files, functions, etc. Each code unit serves only one specific purpose. 
%\item Good naming.
%\item Clear specification of what the code does: comments, tests.
%\item Correctness: testing. 
%\end{itemize}	
%\end{frame}

\begin{frame}[fragile]
\frametitle{Testing}

\begin{itemize}
\item Testing is writing what is expected from certain code and then making sure it works.
\item Automated testing has many benefits:
\begin{itemize}
\item quick testing (we run over 400 tests about every minute)
\item test-driven-development
\begin{itemize}
\item writing tests before code
\item better understand functionality of code
\item make sure code does what we originally expected
\end{itemize}
\end{itemize}
\end{itemize}	
\end{frame}

\begin{frame}[fragile]
\frametitle{Refactoring}
\begin{itemize}
\item Refactoring is changing the structure of code to make it clearer or more efficient  without changing functionality.
\item Frequent refactoring can help keep code more: 
\begin{itemize}
\item readable
\item maintainable
\item flexible
\item scalable
\end{itemize}
\item {\tt preobj} renamed to {\tt message-info-object}
\end{itemize}	
\end{frame}

\begin{frame}[fragile]
\frametitle{Test, refactor, test}
\begin{itemize}
\item Consistently rerunning tests makes it easy for us to see if new code suddenly breaks.
\item Whenever we change anything in the code
\begin{itemize}
\item renaming function names
\item moving around code pieces
\item adding functions
\end{itemize}
we must rerun our tests (by clicking save).
\end{itemize}		
\end{frame}

\begin{frame}[fragile]
\frametitle{What has been accomplished so far}
		
\end{frame}

\section{Modifying a text editor}

\begin{frame}[fragile]
\frametitle{Goal: integrating our software into LightTable}
		
\end{frame}

\end{document} 